\documentclass{article}
\usepackage{gensymb, amsmath, float, graphicx, epstopdf}
\restylefloat{table}
\usepackage[margin=0.75in]{geometry}
\usepackage{verbatim}
\begin{document}

\title{Coasts and Beaches}
\author{HHS Ocean Bowl}
\maketitle

\section{Key Concepts}
	\subsection{Coastlines}
	\begin{itemize}
		\item Coastlines are boundaries where the land and sea interact.
		\item The coastal zone includes open coasts and bays and estuaries, i.e., a coastal zone is both land and water.
		\item The shore is the area from the outer limit of wave action on the bottom, seaward of the lowest tidal reaches, to the limit of the waves’ direct influence on the land.
		\item The beach is an accumulation of sediment (commonly coarser grain sizes such as sands and gravels) that occupies a portion of the shoreline, commonly up to the high tide levels.
		\item Beach areas are not static, and sediments shift and move along, in, and out of beach or shoreline areas constantly.
		\item Coasts are classified as being either erosional or depositional depending on whether they predominantly lose or gain sediment.
		\item Most coasts along the eastern United States are passive coasts and are considered depositional.
		\item Most coasts along the western United States are active coasts and are considered erosional.
	\end{itemize}
	
	\subsection{Types of Coasts}
	\begin{itemize}
		\item Primary coasts are created and maintained by terrestrial or land-based processes. Primary coasts may be formed by:
		\begin{enumerate}
			\item erosion of the land and potential subsidence or sea level rises,
			\item sediments deposited at the shore by rivers, glaciers, or the wind,
			\item volcanic activity, and
			\item vertical movements of the shoreline by tectonic processes.
		\end{enumerate}
		
		\item Secondary coasts are created and maintained by predominantly marine processes. These coasts may be formed by:
		\begin{enumerate}
			\item erosion by waves and currents,
			\item dissolution by seawater,
			\item deposition of sediments by waves, tides, and currents,
			\item erosion, deposition, and binding of sediments and skeletal materials by marine plants and animals.
		\end{enumerate} 
		\item This genetic classification scheme is based on the origin of the dominant processes that shape coastlines, rather than the age or morphology of the coastline.
	\end{itemize}
	
	\subsection{Geological processes}
	\begin{itemize}
		\item Glacial periods tie up seawater in land ice, lowering global sea level. 
		\item Many features of current coastal areas are the direct result of glacial activities. These include erosional effects, such as the cutting of U-shaped river valleys, which may be subsequently flooded to create fjords.
		\item Depositional effects, such as the formation of moraines may form barriers or islands and/or sills that may partially block entrances to fjords.
		\item Indirect glacial effects caused by sea level lowering include exposure of former shelf areas to river erosion and subsequent flooding (when the glaciers melt) to create drowned river or ria coasts, and the isostatic rebound of coastal areas from removal of glacial cover, creating uplifted wave-cut terraces or beach plains.
		\item Many coastlines are completely dominated by sediments brought in by rivers from the interior. It is estimated that rivers carry 530 tons of sediment to the coastal environment each second.  Erosion rates are estimated to be as high as 6 cm (2.4 in) from all land surfaces every 1000 yrs!
		\item Sediment carried to the coast by rivers helps to form and maintain beaches.  Some of this sediment makes its way to the deep-sea floor.
		\item Coastlines characterized by high sediment supply and buffeted by strong prevailing winds are called dune coasts, due to the presence of sand dunes.
		\item Volcanic activity can create flows that reach the sea creating lava coasts; in addition if the crater (i.e., source of the lavas) is located near or on the shoreline and is breached by the sea, it can be called a crater coast.
		\item Tectonic activity along coastal faults can produce a variety of unique shorelines, such as fault bays, and fault coasts.
	\end{itemize}

	\subsection{Secondary Coasts}
	\begin{itemize}
		\item Secondary coasts were once primary coastlines
		\item No matter how irregular a coastline’s morphology is, marine processes constantly reshape it, smoothing its outline or appearance.
		\item Coastlines made of rocks with varying composition and resistance to erosive forces will temporarily have irregular outlines. More resistant rocks will form headlands or points that jut out into the sea. Progressive erosion of these headlands produces sea caves, arches, and sea stacks.
		\item Sea stacks are found in places including northern California, Oregon, Washington, Australia, and New Zealand.
		\item Eroded materials are often removed from the exposed beaches and deposited offshore. Three major types of deposits are commonly found:
		\begin{enumerate}
			\item Bars, which are linear sand deposits paralleling the shore in shallow water,
			\item Barrier islands, which are essentially bars of sand where the sediment supply from the beach and longshore currents has been sufficient to break the surface. These are commonly stabilized by plant growth, but are extremely dynamic and constantly in motion (interference with and building on these islands has proven extremely costly) and,
			\item spits and hooks, which are bar-like deposits connected to the shoreline and often extended across the mouths of bays.  If a spit continues to build offshore and happens to connect with an offshore island, it is called a tombolo.
		\end{enumerate}
		\item Barrier islands along the southeast coast of the United States formed during a period of rising sea level when flooding inundated low-lying coastal areas and isolated the higher dunes at the former coastline.
		\item While barrier islands protect the continental coastline from storm erosion they do so by sustaining damage  themselves.  Severe storms and hurricanes have caused extensive damage and property loss on barrier islands.
		\item One of the great natural disasters in United States history was the 1900 Galveston hurricane.  Galveston sits on Galveston Island, a barrier island along the Texas coast. 
		\item In tropical/subtropical coastal regions plants and animals may bind and trap sediment parallel to the shore, or create massive structures of their own skeletal materials. These systems are generally termed reef coasts.
		\item Low-lying tidal deposits, where sedimentation and plant growth have matched subsidence and recent sea level fluctuations, create great stretches of salt marshes. Salt marshes are extremely productive ecosystems, and many of the world's fisheries depend on their survival.
	\end{itemize}
	
	\subsection{Beaches}
	\begin{itemize}
		\item Beaches can take on many different forms.  The wide variety of features that can be associated with beaches are commonly displayed in a generalized cross section of a beach called a beach profile.
		\item The shore may be subdivided into three major regions called the backshore, foreshore, and offshore.
		\item The offshore region is seaward of the beach.  It includes shallow-water areas seaward of the low tide level out to the limit of wave action on the bottom.  Seasonally mobile sand bars are often found in this region, trending parallel to the shore and separated by intervening troughs.
		\item The combination of the foreshore and backshore are what we call the beach.
		\item The part of the beach that remains exposed above the high tide water level is called the backshore.  At any one season during the year the waves will create a berm on the backshore by cutting a scarp into the beach sediments that is terminated by a berm crest.  The higher energy of winter waves will create a winter berm further up the backshore that will typically remain throughout the year while lower-energy summer waves create summer berms closer to the water's edge that will be erased by the next winter's storm waves.
		\item Two major features that we may see in the foreshore are low tide terraces cut by wave action during the low tide and a steeper slope extending up to the exposed part of the beach, called the beach face, which extends from the low to the high tide water levels including the swash zone.
		\item Some shorelines do not have visible beaches, instead they may simply have cliff faces.
	\end{itemize}
\end{document}